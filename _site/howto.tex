
\font\chant opusg20

\centerline{\bigtype How To Sing Chant}

\doubleline

\bigskip

Like books on how to speak French, it is best if you can learn by 
listening.  Since that is not always possible here are a few pointers
to help you get started.

\bigskip

Chant is written on a four line staff.  That gives enough room for a 
comfortable vocal range.  The melodies can be sung at whatever pitch
suits the choir, usually around speaking pitch or higher.

The \kern2pt\raise.3em\hbox{\chant\char8}\kern-2pt clef marks the line for the `Do' note.  
The \kern2pt\raise1.8em\hbox{\chant\char9}\kern-2pt clef 
marks the line for the `Fa' note.  The notes are read left to right
with a steady rhythm following the natural rise and fall of the text.

\bigskip

Here is an example.

\setgregorian1
\setclef12
\musicindent14mm
\raisesong3\Internote
\initiumgregorianum
\musicinitial{}{A}%
\sgn {}{-}{S}\pes Na\egn
\sgn p{\'e}{r-}\climacus cba\egn
\sgn ges\pes bc\egn
\spatium
\sgn me,\punctum d\augmentum d\egn
\spatium
\Finisgregoriana

\bigskip

This neat, compact excerpt could be written like this:

{
\resetlayout

\def\Notes{\vnotes4}
\def\NOtes{\vnotes5}
\def\NOTes{\vnotes6}


%\generalmeter\meterC
%\belowsystem{6}
\smallmusicsize
\setclef1{\treble}
\startpiece
\Notes\cchar{-6}{As-}\ibu1{g}{2}\qb1{g}\en
\Notes\tb1\qb1{'a}\en
\Notes\cchar{-6}{p\'er-}\ibu1{'c}{-1}\qb1{'c}\en
\Notes\qb1{'b}\en
\Notes\tb1\qb1{'a}\en
\Notes\cchar{-6}{ges}\ibu1{'b}{2}\qb1{'b}\en
\Notes\tb1\qb1{'c}\en
\NOtes\cchar{-6}{me,}\ha{'d}\en
\endpiece

\bigskip

Or a little lower like this:

%\generalmeter\meterC
%\belowsystem{6}
\generalsignature{+1}
\smallmusicsize
\setclef1{\treble}
\startpiece
\Notes\cchar{-6}{As-}\ibu1{d}{2}\qb1{d}\en
\Notes\tb1\qb1{e}\en
\Notes\cchar{-6}{p\'er-}\ibu1{g}{-1}\qb1{g}\en
\Notes\qb1{f}\en
\Notes\tb1\qb1{e}\en
\Notes\cchar{-6}{ges}\ibu1{f}{2}\qb1{f}\en
\Notes\tb1\qb1{g}\en
\NOtes\cchar{-6}{me,}\ha{'a}\en
\endpiece


}

\bigskip

Can you see the advantage of the first notation?  The text gives the
spacing of the notes rather than being chopped into syllables and
spread according to time.

\bigskip

A few notes are a bit puzzling.  The {\chant\char16\raise2em\hbox{\char15}}
~is a pair of notes sung bottom to top.  
The {\chant\raise1em\hbox{\char30}\kern2em\raise2em\hbox{\char15}}
~is a trio of three notes, the first two at either end of the swoosh and
finishing with the last note up on the stick.

\bigskip

Dots and dashes roughly double the time the note is held.
The volume swells according to the words.  Accent marks show
where to stress the syllable.  Latin never stresses the last 
syllable so two syllable words are not marked to save ink.
The voice dies down at the end of a phrase.  

\bigskip

The different barlines
indicate different separations or pauses.  The bigger the line the
bigger the pause.  The smaller divisions do not warrant a breath.
Try to get a whole sentence in one breath, or at least avoid taking
a breath in the middle of a word.

\bigskip

There are a few unusual notes to look for.  Some say the  \kern1pt{\chant\raise1em\hbox{\char17}}
~ was once sung with a wobble.  It is tricky
to get a choir to wobble all together so now most agree to emphasize the
previous note and skip lightly over the \kern1pt{\chant\raise1em\hbox{\char17}}~.
Sometimes notes at the end of a bunch are shrunk to show how the sound 
diminishes.

