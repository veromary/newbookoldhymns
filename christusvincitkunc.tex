
{\noindent \itlyrics Alternate setting by A. Kunc:}

\medskip


%\beginhymn Christus vincit.

\begingroup

%\moveleft5mm\vbox{
%\hsize100mm

\resetlayout

\def\notes{\vnotes3}%quavers
\def\Notes{\vnotes4}%minims
\def\NOtes{\vnotes5}%dotted minims
\def\NOTes{\vnotes6}%semibreves

%\rmlyrics

\def\engl#1{\cchar{16}{\itlyrics #1}}
\def\lati#1{\cchar{22}{\rmlyrics #1}}

\def\vengl#1{\cchar{-13}{\itlyrics #1}}
\def\vlati#1{\cchar{-7}{\rmlyrics #1}}

\instrumentnumber2
%\setstaffs1{2}
\setclef1{\bass}
\smallmusicsize
\interstaff{16}
\sepbarrules
\group{12}
%\generalmeter{\meterfrac42}
%\group{12} % don't know why I had this twice
\startpiece
\notes\en
\Notes\lati{Chri-}\hd{M}\hu{c}|\hd{f}\hu{'a}\en
\Notes\lati{stus}\hd{M}\hu{c}|\hd{f}\hu{'a}\en
\Notes\lati{vin-}\hd{L}\hu{c}|\hd{g}\hu{'c}\en
\Notes\lati{cit}\hd{L}\hu{c}|\hd{g}\hu{'c}\en
\setdoublebar\endpiece

%}

\endgroup

\interstaff9

