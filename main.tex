\documentclass[11pt]{article} % use larger type; default would be 10pt

% usual packages loading:
\usepackage{fontspec}
%\setmainfont{Linux Libertine O}
\setmainfont{Calluna}
%\newfontface{\vectis}{EBH Alternates}
%\newfontface{\vectis}{Beowulf Modern}
%\newfontface{\vectis}{Michaelmas}
\newfontface{\vectis}{Vinque Regular}
\newfontface{\lombard}{EasyLombardic Two}
\newfontface{\inglesh}{Calluna Sans}
\usepackage[autocompile]{gregoriotex} % for gregorio score inclusion
\usepackage{multicol}
%>>>>>>> Stashed changes
\usepackage{graphicx} % support the \includegraphics command and options
\usepackage{geometry} % See geometry.pdf to learn the layout options. There are lots.
\geometry{a5paper} % or letterpaper (US) or a5paper or....
\usepackage[cm]{fullpage} % to reduce the margins

%<<<<<<< Updated upstream
% to change the font to something better, you can install the kpfonts package (if not already installed). To do so
% go open the "TeX Live Manager" in the Menu Start->All Programs->TeX Live 2010
%=======
\tolerance=1000
\pretolerance=800

\grechangestyle{translation}{\inglesh \fontsize{10}{11}\selectfont}
\newcommand\translationfont[1]{{\inglesh \fontsize{10}{11}\selectfont #1}}

\grechangestyle{initial}{\lombard \fontsize{38}{38}\selectfont}
\grechangestyle{annotation}{\tiny}
%>>>>>>> Stashed changes


% here we begin the document
\begin{document}

% We set red lines here, comment it if you want black ones.
\gresetlinecolor{gregoriocolor}


~

\bigskip

~

\begin{center}

\textsc{\huge A New Book}

\bigskip



\textsc{\huge of Old Hymns}

\bigskip


\textsc{\large in Latin and English}

\bigskip


chant

polyphony

rounds

a litany

for all the seasons

of the Church's year

\end{center}

\bigskip

\eject

~

\vfill

First Edition July 2004
Second Edition January 2006
Third Edition October 2007

Compiled by Veronica Brandt http://brandt.id.au.

\eject

\tableofcontents

\gregorioscore{gabc/va--oremus_pro_pontifice--solesmes}

\section{Advent}

\gregorioscore{gabc/hy--conditor_alme_siderum-v1--solesme}

Veni veni Emmanuel - need to type it up in Lilypond or something like that.

or with all the verses

\subsection{Conditor Alme Siderum}

\gregorioscore{gabc/hy--conditor_alme_siderum--solesmes}

\subsection{Veni O Sapientia}
>>>>>>> Stashed changes

\gregorioscore{gabc/va--rorate_caeli--solesmes.1}

\gregorioscore{gabc/va--rorate_caeli--solesmes.2}

\gregorioscore{gabc/va--rorate_caeli--solesmes.3}

\gregorioscore{gabc/va--rorate_caeli--solesmes.4}

\gregorioscore{gabc/va--rorate_caeli--solesmes.5}

\section{Christmas}

\gregorioscore{gabc/hy--puer_natus_in_bethlehem--solesmes}

\section{Holy Name}

\gregorioscore{gabc/hy--jesu_dulcis_memoria--solesmes}

\gregorioscore{gabc/an--lumen_ad_revelationem--solesmes}

\section{Epiphany}

\section{Candlemas}

\section{Lent}

\gregorioscore{gabc/va--attende_domine--solesmes.1}

\gregorioscore{gabc/va--attende_domine--solesmes.2}

\gregorioscore{gabc/an--parce_domine--solesmes}

\section{Passiontide}

%Gloria Laus

\gregorioscore{gabc/hy--gloria_laus--solesmes.1}


\gregorioscore{gabc/hy--gloria_laus--solesmes.2}


\gregorioscore{gabc/hy--gloria_laus--solesmes.3}


\gregorioscore{gabc/hy--gloria_laus--solesmes.4}


\gregorioscore{gabc/hy--gloria_laus--solesmes.5}


\gregorioscore{gabc/hy--gloria_laus--solesmes.6}

\gregorioscore{gabc/hy--vexilla_regis_prodeunt--solesmes.1verse}


\section{Easter}

\section{Ascension}

\section{Pentecost}

\section{Trinity}

\section{Sacred Heart}

\section{Corpus Christi}

\section{Christ the King}

\section{All Saints}

\section{All Souls}

\section{Marian}

\gregorioscore{gabc/an--ave_regina_caelorum--simplex}

\section{For Peace}

% The title:
%\begin{center}\begin{huge}\textsc{For Peace}\end{huge}\end{center}

% We set VII above the initial.
%\gresetfirstlineaboveinitial{\small \textsc{\textbf{II}}}{\small \textsc{\textbf{II}}}

% and finally we include the score. The file must be in the same directory as this one.
\gregorioscore{gabc/an--da_pacem_domine--solesmes.tex}

\bigskip


\section{Thanksgiving}

\section{Kyriale}

\section{Benediction}

\section{Index}

\end{document}
