% !TEX TS-program = lualatex
% !TEX encoding = UTF-8

% This is a simple template for a LuaLaTeX document using gregorio scores.

\documentclass[11pt]{article} % use larger type; default would be 10pt

% usual packages loading:
\usepackage{fontspec}
\setmainfont{EB Garamond}
\usepackage{graphicx} % support the \includegraphics command and options
\usepackage{geometry} % See geometry.pdf to learn the layout options. There are lots.
\geometry{a5paper} % or letterpaper (US) or a5paper or....
\usepackage{gregoriotex} % for gregorio score inclusion
\usepackage[cm]{fullpage} % to reduce the margins
\usepackage{multicol}

% to change the font to something better, you can install the kpfonts package (if not already installed). To do so
% go open the "TeX Live Manager" in the Menu Start->All Programs->TeX Live 2010

\let\oldtextbf\textbf

\renewcommand\textbf[1]{%
    \pdfliteral direct {2 Tr 0.2 w}%
     \oldtextbf{#1}%
    \pdfliteral direct {0 Tr 0 w}%
}
\newcommand\embolden[1]{%
    \pdfliteral direct {2 Tr 0.2 w}%
     #1%
    \pdfliteral direct {0 Tr 0 w}%
}
\renewcommand\greboldfont[1]{%
    \pdfliteral direct {2 Tr 0.2 w}%
    \oldtextbf{#1}%
    \pdfliteral direct {0 Tr 0 w}%
}


% here we begin the document
\begin{document}

% Here we set the space around the initial.
% Please report to http://home.gna.org/gregorio/gregoriotex/details for more details and options
\setspaceafterinitial{2.2mm plus 0em minus 0em}
\setspacebeforeinitial{2.2mm plus 0em minus 0em}

% Here we set the initial font. Change 43 if you want a bigger initial.
\def\greinitialformat#1{%
{\fontsize{43}{43}\selectfont #1}%
}

% We set red lines here, comment it if you want black ones.
\redlines


\includescore{gabc/va--oremus_pro_pontifice--solesmes}

\section{Advent}

\includescore{gabc/hy--conditor_alme_siderum-v1--solesme}

Veni veni Emmanuel - need to type it up in Lilypond or something like that.


\includescore{gabc/va--rorate_caeli--solesmes.1}

\includescore{gabc/va--rorate_caeli--solesmes.2}

\includescore{gabc/va--rorate_caeli--solesmes.3}

\includescore{gabc/va--rorate_caeli--solesmes.4}

\includescore{gabc/va--rorate_caeli--solesmes.5}

\section{Christmas}

\section{Holy Name}

\includescore{gabc/hy--jesu_dulcis_memoria--solesmes}

\includescore{gabc/an--lumen_ad_revelationem--solesmes}

\section{Epiphany}

\section{Candlemas}

\section{Lent}

\includescore{gabc/an--parce_domine--solesmes}

\section{Passiontide}

\includescore{gabc/hy--vexilla_regis_prodeunt--solesmes}

\section{Easter}

\section{Ascension}

\section{Pentecost}

\section{Trinity}

\section{Sacred Heart}

\section{Corpus Christi}

\section{Christ the King}

\section{All Saints}

\section{All Souls}

\section{Marian}

\section{For Peace}

\% The title:
\begin{center}\begin{huge}\textsc{For Peace}\end{huge}\end{center}

% We set VII above the initial.
\gresetfirstlineaboveinitial{\small \textsc{\textbf{II}}}{\small \textsc{\textbf{II}}}

% and finally we include the score. The file must be in the same directory as this one.
\includescore{gabc/an--da_pacem_domine--solesmes.tex}

\bigskip


\section{Thanksgiving}

\section{Kyriale}

\section{Benediction}

\section{Index}

\end{document}
